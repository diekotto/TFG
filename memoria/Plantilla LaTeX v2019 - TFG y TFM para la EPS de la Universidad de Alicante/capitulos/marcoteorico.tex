%%%%%%%%%%%%%%%%%%%%%%%%%%%%%%%%%%%%%%%%%%%%%%%%%%%%%%%%%%%%%%%%%%%%%%%%
% Plantilla TFG/TFM
% Escuela Politécnica Superior de la Universidad de Alicante
% Realizado por: Jose Manuel Requena Plens
% Contacto: info@jmrplens.com / Telegram:@jmrplens
%%%%%%%%%%%%%%%%%%%%%%%%%%%%%%%%%%%%%%%%%%%%%%%%%%%%%%%%%%%%%%%%%%%%%%%%

\chapter{Estado de arte}
En este apartado se va a presentar el impacto y el beneficio de la tecnología entrando, cada vez más, en el campo práctico de los servicios sociales.
\vspace{1em}
\par La tecnología ha revolucionado nuestra forma de consumir, de relacionarnos y de informarnos. Donde tampoco han sido ajenos a la evolución de la tecnología, es en el ámbito de los servicios sociales. Sin embargo, comparándolo con otros sectores mucho más maduros y con más presupuesto como la banca o el comercio, el sector social parece no haber sido capaz de adoptar o tener acceso a toda la tecnología que ya está desarrollada, ésta podría facilitar la sostenibilidad de los servicios, la autonomía y la eficiencia en el servicio humano que se intenta dar.
\vspace{1em}
\par 