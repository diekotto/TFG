%%%%%%%%%%%%%%%%%%%%%%%%%%%%%%%%%%%%%%%%%%%%%%%%%%%%%%%%%%%%%%%%%%%%%%%%
% Plantilla TFG/TFM
% Escuela Politécnica Superior de la Universidad de Alicante
% Realizado por: Jose Manuel Requena Plens
% Contacto: info@jmrplens.com / Telegram:@jmrplens
%%%%%%%%%%%%%%%%%%%%%%%%%%%%%%%%%%%%%%%%%%%%%%%%%%%%%%%%%%%%%%%%%%%%%%%%

\chapter{Conclusiones}
\label{conclusiones}
Una vez finalizado el proyecto, no está de mas recapitular y revisar el trabajo realizado. Es una buena forma de revisar si se han logrado cumplir los objetivos que se marcaron en un inicio.
\section{Revisión de los objetivos marcados}
Los objetivos que se pretendían conseguir son los siguientes:
\begin{itemize}
    \item Investigar y analizar soluciones hardware lowcost
    \begin{itemize}
        \item Se ha realizado un estudio de diferentes tipos de sistemas a tener en cuenta, pasando desde el self-hosting con una raspberry pi a terminar decidiendo usar servicios serverless cloud en heroku.
    \end{itemize}
    \item Investigar y analizar el estado de los bancos de alimentos en el mundo y su relación con la tecnología
    \begin{itemize}
        \item Se han investigado proyectos y necesidades, quedando patente las potenciales implicaciones positivas que tendría adoptar en mayor medida la tecnología en el campo social
    \end{itemize}
    \item Estudiar el mercado para decidir el stack tecnológico en el que desarrollar la aplicación
    \begin{itemize}
        \item Se ha realizado un estudio centrado en encuestas de StackOverflow, líder en el de la comunidad online tecnológica, para decidir lenguajes a utilizar.
    \end{itemize}
    \item Estudiar alternativas de software en el mercado
    \begin{itemize}
        \item Se ha investigado si la solución que se pretende con este proyecto podría ser solventada por uno de los CRM que ya existen en el mercado, si bien hay mucha comunidad tecnológica a penas hay proyectos libres adhoc para este tipo de necesidades.
    \end{itemize}
    \item Diseñar la aplicación web conforme a los requisitos recogidos del economato social
    \begin{itemize}
        \item Se estudiaron y documentaron los requisitos fundamentales que la aplicación debería cubrir y no sólo se han cumplido en su totalidad si no que se han implementado mejoras a lo largo del proyecto y, cuando ya se había finalizado el desarrollo, se aceptaron dos nuevas implementaciones y se llevaron a cabo en tiempo y forma.
    \end{itemize}
    \item Uso de metodología ágil con entrega continua
    \begin{itemize}
        \item Se ha adoptado con éxito una metodología ágil basada en sprints de 3 semanas, con reuniones en vivo con el economato para recoger feedback y requisitos; habiendo construido así una solución totalmente a medida.
    \end{itemize}
    \item Entregar la aplicación web en su versión acabada
    \begin{itemize}
        \item Se ha entregado la aplicación finalizada y usable desplegada en un host gratuito, además de guía de uso y vídeo tutoriales para facilitar la adopción de los voluntarios del economato social.
    \end{itemize}
\end{itemize}
\section{Conclusiones}
Para concluir esta memoria, cabe destacar que ha sido un auténtico placer desarrollar una aplicación de estas características debido al uso que se le pretende dar. Ha sido un desafío, personal y académico, tanto por el alcance del proyecto como por la situación de pandemia en la que nos encontramos. Haber sido parte de un proyecto como este hace mella en uno, saber que los conocimientos y habilidades pueden ser puestas en pos de los más necesitados y de que sólo hacen falta ganas y no grandes presupuestos para mejorar la calidad laboral de aquellos que se ponen en primera línea para ayudar a los más necesitados. Ha sido una experiencia enriquecedora y si puedo, seguiré colaborando y mejorando este proyecto a lo largo del tiempo hasta que, por la razón que sea, no me sea posible aportar nada más. Deseo que la adopción de la tecnología en el campo social no deje de crecer y que, efectivamente, algún día los bancos sociales tengan voluntarios o financiación suficiente como para que la tecnología deje de ser un muro tan insalvable como lo es hoy.
\vspace{1em}
\par Además de lo personal, profesionalmente ha sido un reto. He tenido que desempolvar conocimientos de la carrera, sacar las mejores herramientas y dar lo mejor de mí para llegar a buen puerto. No es mi primera experiencia como full-stack pero sí puedo decir sin miedo que es la primera vez que desarrollo yo sólo un proyecto de éste alcance y del que hay gente que espera tanto, me he visto necesitado de sacar a relucir patrones de diseño y conocimientos avanzados en tecnología y arquitectura cloud que he adquirido personal y profesionalmente en mis años de experiencia. Siento que ha sido una gran experiencia y que, aunque esté terminando de escribir la memoria de este proyecto, aún queda mucho por desarrollar y mejorar en la entrega que terminar aquí supone.
\vspace{1em}
\par Al principio no las tenía todas conmigo en la decisión de volcar los esfuerzos en desarrollar una solución cloud native, dado que la idea de usar el self-hosting tenía mucho peso inicial. Pero por suerte no hubo arrepentimientos, las automatizaciones hicieron su magia para facilitarme el trabajo en el CI/CD y las capas gratuitas de hosting y base de datos nos han permitido entregar un producto de calidad que funciona a coste cero y que, dada la naturaleza inicial de configuración y desarrollo, podríamos migrar a cualquier otro servicio cloud o proveedor de base de datos con muy poco esfuerzo; en caso de ser necesario.
\vspace{1em}
\par Para finalizar me gustaría comentar que aún hay mucho por recorrer en la tecnología centrada en el campo social y, con los retos que tenemos encima ahora mismo por la pandemia, seguro que hay muchísimas ideas que podemos llevar a cabo. Desarrollos y avances que no pueden hacer otra cosa que mejorar nuestra sociedad justo donde más falta hace, donde no hay recursos informáticos ni educación tecnológica.